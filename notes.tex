\documentclass[11pt,a4paper]{report}
\usepackage[utf8]{inputenc}
\usepackage[english]{babel}
\usepackage{amsmath}
\usepackage{amsfonts}
\usepackage{amssymb}
\usepackage{graphicx}

\title{Physics For Electronic Engineers Revision Notes}
\author{Tom Eccles}

\begin{document}
\maketitle
\pagebreak
\tableofcontents
\pagebreak
\chapter{Electromagnetism}
\section{Basics}
\begin{itemize}
\item Charge is quantised in units of $e$
\item Charge is conserved
\end{itemize}

\subsection{Coulomb's Law}
\begin{equation*}
	\vec{F} = k\frac{q_1q_2}{\vec{r}^2}
\end{equation*}

Where $k$ is the electrostatic constant:
\begin{equation*}
	k = \frac{1}{4\pi\varepsilon_0}
\end{equation*}

This is very similar to Newton's law for gravitation:
\begin{equation*}
	\vec{F}=G\frac{m_1m_2}{\vec{r}^2}
\end{equation*}

The total force on a particle is equal to the sum of all the forces from all other particles. \textit{Make sure you get the vector algebra correct when doing this.}

\section{Electric Fields}
Electric field is defined as the force per unit charge:
\begin{equation*}
	\vec{E} = \frac{\vec{F}}{q} = \frac{Q}{4\pi\varepsilon_0 \vec{r}^2}
\end{equation*}

The electric field is often drawn as arrows on a diagram (from the perspective of the force upon a small positive test charge).

As with force, the total electric field equals the sum of all electric fields in a system (again, being careful with the vector algebra).

\subsection{Example: Electric Field From a Charged Ring}
Suppose there is a positively charged ring in the XY-plane, centred about the origin and with an average radius $R$ with a thickness far less than R. Assuming that the charge on the ring is far greater than $e$ (so that charge can be treated as continuous) the electric field at a point P distance $z$ along the Z-axis can be found by the following:

As previously discussed, the total electric field is the sum of all of the fields from all of the charges and so:
\begin{equation*}
\vec{E_T} = \int_0^{E_T}\, dE
\end{equation*}
Where $dE$ is an infinitesimally small section of the ring. Considering the electric field in the Z direction, $dE$ will be
\begin{eqnarray*}
	dE_z = \frac{\delta Q}{4\pi\varepsilon_0(R^2+z^2)}\cdot\frac{z}{\sqrt{R^2+z^2}} = \frac{\delta Qz}{4\pi\varepsilon_0(R^2+z^2)^{\frac{3}{2}}} \\
	\therefore E_z = \frac{Qz}{4\pi\varepsilon_0(R^2+z^2)^{\frac{3}{2}}}
\end{eqnarray*}

As point P is defined to be along the axis perpendicular to the ring's centre, the X and Y components of the field must cancel out due to symmetry. Therefore the whole field is:
\begin{equation*}
	E = \frac{Qz}{4\pi\varepsilon_0(R^2+z^2)^{\frac{3}{2}}}
\end{equation*}

\section{Electric Potential}
\begin{itemize}
	\item This is a scalar quantity and so is path independent
	\item Potential is measured relative to the potential somewhere else
	\item Often measured relative to the potential at $r=\infty$ because $V_\infty=0$
	\item I find it useful to think of electric potential as the potential energy per unit charge
\end{itemize}
\begin{equation*}
	V = - \int_i^f E(x)\delta x
\end{equation*}

\subsection{Example: Single Charge}
The potential at distance $R$ from charge $Q$:
\begin{equation*}
	V = -\int_{\infty}^R\frac{Q}{4\pi\varepsilon_0r^2}dr = \frac{Q}{4\pi\varepsilon_0}\left(\frac{1}{R} - \frac{1}{\infty}\right) = \frac{Q}{4\pi\varepsilon_0R}
\end{equation*}
Observe that the maths breaks down when very close to $R=0$. Fortunately real charged particles do have size (although it is incredibly small).

\subsection{Electric Potential Energy}
\begin{itemize}
	\item Analogous to electric potential energy
	\item Scalar - therefore path independent
	\item $E_p = qV$
\end{itemize}

\subsection{Equipotential Lines}
\begin{itemize}
	\item Equipotential Lines are always perpendicular to field lines
	\item These are like the height lines on a map.
\end{itemize}

\subsection{Spherical Conductor}
Imagine a spherical conductor with charge evenly distributed across it:
\begin{itemize}
	\item There is no electric field inside the conductor because of the symmetry of the sphere: the super positions cancel out.
	\item Therefore inside the sphere the electric potential is constant.
	\item Newton's shell theorem\footnote{See https://en.wikipedia.org/wiki/Shell\_theorem} also tells us that from outside the shell we can treat the spherical conductor like a particle at the centre of the sphere.
\end{itemize}

\chapter{Devices}
\section{Resistors}
\subsection{Basics}
Resistance is defined using Ohm's Law:
\[R=\frac{V}{i}\] Note that the resistance is a property of a material and so does not depend upon potential difference and current. The resistance of a device can be calculated by the following:
\[R=\rho\frac{l}{S}\] Where $\rho$ is the resistivity of the material, $l$ is the length and $S$ is it's surface area.

Most real components have resistance however this is often unintended and in simple models circuits are modelled with only resistance in intentionally placed resistors.

The unit of resistance is the Ohm ($\Omega$). As can be seen from Ohm's Law, one $\Omega$ is equivalent to one $VA^{-1}$.

\subsection{Derivation Of Ohm's Law}
\subsubsection{Current and Current Density}
Current is the rate of flow of charge:
\[i=\frac{dq}{dt}\]

Current density for constant current and area: 
\[J = \frac{I}{A}\]

Current density for variable current: 
\[i = \int \vec{J} \cdot d\vec{A}\] 

\subsubsection{Drift Velocity}
For all electric fields:
\[F = ma = qE\]
\[\therefore a = \frac{eE}{m}\]

However in materials with resistance (not superconductors) the acceleration is limited because the electrons collide with ions in the material. This scatters the electrons and stops them accelerating for too long.

Therefore the drift velocity of electrons due to an electric field through a material should be calculated using a statistical average using the average time between collisions $\tau$:
\[v_d = a\tau = \frac{eE\tau}{m}.\]

The drift velocity is much less than the thermal/diffusion velocity. 

\subsubsection{Using this to get current}
Total mobile charge in material length $d$ with surface area $A$ and $n$ charges $e$ per unit volume:
\[Q=neAd\]

The time for this mobile charge to sweep past the measuring point (ignoring acceleration) is:
\[t=\frac{d}{v_d}[\]
\[\therefore I = \frac{Q}{t} = neAv_d\]

\subsubsection{Ohm's Law (microscopic)}
Using previous equations for $I$, $J$ and $v_d$:
\[J = env_d=\frac{ne^2E}{m}\tau\]

Another definition for current density (using conductivity $\sigma$):
\[J=\sigma E\]
\[\therefore \sigma = \frac{ne^2\tau}{m}\]

Resistivity is the inverse of conductivity:
\[\rho = \frac{1}{\sigma}\]

From before:
\[i=JA\]

Assume uniform electric field:
\[V=EL\]

Using the 'resistivity equation':
\[i=\frac{AV}{\rho L}\]
\[\therefore R=\frac{V}{i}\]

\subsubsection{What does this tell us?}
In symmetric non-superconductors  $I \propto V$. Materials which obey this are called ohmic conductors.

\subsubsection{Mobility}
Mobility characterises how quickly electrons or holes move through a conductor when an electric field is applied. 

Electron mobility $\mu$ is defined as
\[\mu = \frac{ne^2\tau}{m} = \frac{\sigma}{ne}\]

Which means it is the conductivity per unit charge.

\subsection{Power}
Power is defined as the rate of energy transfer:
\[P_{tr} = \frac{dW}{dt} = \frac{dW}{dq}\frac{dq}{dt} = vi\].

By substituting in from Ohm's Law we can see that:
\[P_{loss} = i^2R = \frac{v^2}{R}\]

Note that the $v$ in Ohm's Law is the potential across the resistor, not the potential of the resistor relative to an arbitrary ground. This is why very high voltages are used in power lines: $\Delta v << v$ (because $\Delta v$ and $v$ are measured using different reference points) and so this leads to very high efficiency compared to when higher currents and lower potentials are used.

\subsubsection{Heat}
Following on from the definition of power, the thermal energy released by the resistor in time $t = t_2 - t_1$ can be calculated by the following (thermodynamics states that all of the energy dissipated will eventually become heat):
\[W=\int_{t_1}^{t_2} p(t) \: dt = \int_{t_1}^{t_2} v(t)i(t) \: dt\]

Of-course, real materials have a limited amount of power which they can dissipate without becoming damaged. For resistors this is called the resistor power rating. Many resistors are typically rated at $1/10$, $1/8$ or $1/4$ Watts. Resistors with higher power ratings than this are called power resistors.

For similar reasons, resistors also have temperature ratings which depend upon the power loads on the resistors.

The resistance from the material will also change with temperature. This is described by the temperature co-efficient of resistance (TCR) which is expressed as the change in resistance (in parts per million) per degree Kelvin. This is not typically linear, instead being parabolic with temperature.

Sometimes resistance can change with potential as well and so there is a voltage rating and coefficient analogous to those for temperature. The voltage rating can be calculated as follows:
\[Rated Voltage(V)=\sqrt{Rated Power(W) \times Nominal Resistance Value(\Omega)}\]

\subsection{Noise in a Resistor}
In general thermal noise dominates at high frequencies and current noise at low frequencies.
\subsubsection{Thermal Noise}
Thermal noise is evenly distributed across all frequencies and is dependent only on the value of resistance and the resistor's temperature:
\[S_T = 4kTR\]
Where $k = 1.3807 \times 10^{-23} JK^{-1}$ (Boltzmann's constant). 

\subsubsection{Current Noise}
Current noise in a resistor does depend upon the material and the frequency:
\[S_E = C \, \frac{U^2}{f}\] where $C$ is a constant depending upon the material of the resistor and $U$ is the potential drop across the resistor.
This is often expressed in decibels:
\[[NI]_{dB} = 20 \log \left( \frac{u}{U} \times 10^6 \right)\]

\section{Polarisation}

\section{Capacitors}
A capacitor is a device which stores charge. It is made of two conductors separated by an insulator.

The capacitance of a capacitor is defined as the ratio of charge to potential: however capacitance is a constant and is determined by neither charge nor potential.
\[\frac{Q}{V}\]

Instead the capacitance of a given capacitor depends upon its dimensions and the permittivity of it's dielectric.
\[C=\varepsilon \varepsilon_0 \frac{S}{d}\]

Where $\varepsilon_0 = 8.85 \times10^{-12} \: Fm^{-1}$ is the permittivity of a vacuum, $S$ is the surface area of the conducting plates and $d$ is their separation.

\subsection{Energy}
Remember from before that 
\[V=\frac{dW}{dq}\]
\[\therefore q=C\frac{dW}{dq}\]
\[\int_0^Q q \, dq = C \int_0^Q dW\]
\[\frac{1}{2} \, Q^2 = CW\]
\[\therefore W=\frac{1}{2}CV^2=\frac{1}{2}QV\]

\subsection{Imperfect Dielectrics and Complex Permittivity}
In an ideal dielectric there is no energy loss however in real dielectrics defects and impurities lead to various charge carriers in dielectrics. Under the influence of an electric field, current flows through the dielectrics: causing it to act like a resistor. 

Also, material's polarisation do not respond instantaneously to changes in electric field and so the response of a capacitor greatly depends upon the frequency applied to it. This introduces a phase difference to the response: motivating the use of complex numbers to model this behaviour.

This real behaviour can be modelled by imagining a capacitor where the permittivity is a complex number: $\varepsilon^*$, which is called complex permittivity. Let $Re(\varepsilon^*)=\varepsilon'$ and $Im(\varepsilon^*)=\varepsilon''$. To model the $C \, F$ capacitor we will represent it as an ideal capacitor $C_p$ with capacitance $C_0 =\varepsilon'\varepsilon_0\frac{S}{d}$, connected in parallel across an ideal resistor $R_p$ with resistance $\varepsilon'' \, \Omega$.

As the permittivity is a complex number, the current though the capacitor will also be complex. Therefore an Argand diagram can be drawn with current on the imaginary axis and potential on the real axis. The current response from the capacitor for different potentials for a given angular velocity $\omega$ can be plotted as s vector on the Argand diagram. The angle between the imaginary axis (current) and this vector is the phase shift described previously.

\section{Electrets}

\chapter{Oscillations and Waves}

\chapter{Batteries and Fuel Cell}

\chapter{Appencices}
\section{Distrubution and Licensing}
This work is distrubuted under Creative Commons Attribution-ShareAlike 4.0
International (see https://creativecommons.org/by-sa/4.0/).

\end{document}
