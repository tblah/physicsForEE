\documentclass[11pt,a4paper]{report}
\usepackage[utf8]{inputenc}
\usepackage[english]{babel}
\usepackage{amsmath}
\usepackage{amsfonts}
\usepackage{amssymb}
\usepackage{graphicx}

\title{Physics For Electronic Engineers Revision Notes}
\author{Tom Eccles}

\begin{document}
\maketitle
\pagebreak
\tableofcontents
\pagebreak
\chapter{Electromagnetism}

\chapter{Devices}
\section{Resistors}
\subsection{Basics}
Resistance is defined using Ohm's Law:
\[R=\frac{V}{i}\] Note that the resistance is a property of a material and so does not depend upon potential difference and current. The resistance of a device can be calculated by the following:
\[R=\rho\frac{l}{S}\] Where $\rho$ is the resistivity of the material, $l$ is the length and $S$ is it's surface area.

Most real components have resistance however this is often unintended and in simple models circuits are modelled with only resistance in intentionally placed resistors.

The unit of resistance is the Ohm ($\Omega$). As can be seen from Ohm's Law, one $\Omega$ is equivalent to one $VA^{-1}$.

\subsection{Derivation Of Ohm's Law}
\subsubsection{Current and Current Density}
Current is the rate of flow of charge:
\[i=\frac{dq}{dt}\]

Current density for constant current and area: 
\[J = \frac{I}{A}\]

Current density for variable current: 
\[i = \int \vec{J} \cdot d\vec{A}\] 

\subsubsection{Drift Velocity}
For all electric fields:
\[F = ma = qE\]
\[\therefore a = \frac{eE}{m}\]

However in materials with resistance (not superconductors) the acceleration is limited because the electrons collide with ions in the material. This scatters the electrons and stops them accelerating for too long.

Therefore the drift velocity of electrons due to an electric field through a material should be calculated using a statistical average using the average time between collisions $\tau$:
\[v_d = a\tau = \frac{eE\tau}{m}.\]

The drift velocity is much less than the thermal/diffusion velocity. 

\subsubsection{Using this to get current}
Total mobile charge in material length $d$ with surface area $A$ and $n$ charges $e$ per unit volume:
\[Q=neAd\]

The time for this mobile charge to sweep past the measuring point (ignoring acceleration) is:
\[t=\frac{d}{v_d}[\]
\[\therefore I = \frac{Q}{t} = neAv_d\]

\subsubsection{Ohm's Law (microscopic)}
Using previous equations for $I$, $J$ and $v_d$:
\[J = env_d=\frac{ne^2E}{m}\tau\]

Another definition for current density (using conductivity $\sigma$):
\[J=\sigma E\]
\[\therefore \sigma = \frac{ne^2\tau}{m}\]

Resistivity is the inverse of conductivity:
\[\rho = \frac{1}{\sigma}\]

From before:
\[i=JA\]

Assume uniform electric field:
\[V=EL\]

Using the 'resistivity equation':
\[i=\frac{AV}{\rho L}\]
\[\therefore R=\frac{V}{i}\]

\subsubsection{What does this tell us?}
In symmetric non-superconductors  $I \propto V$. Materials which obey this are called ohmic conductors.

\subsubsection{Mobility}
Mobility characterises how quickly electrons or holes move through a conductor when an electric field is applied. 

Electron mobility $\mu$ is defined as
\[\mu = \frac{ne^2\tau}{m} = \frac{\sigma}{ne}\]

Which means it is the conductivity per unit charge.

\subsection{Power}
Power is defined as the rate of energy transfer:
\[P_{tr} = \frac{dW}{dt} = \frac{dW}{dq}\frac{dq}{dt} = vi\].

By substituting in from Ohm's Law we can see that:
\[P_{loss} = i^2R = \frac{v^2}{R}\]

Note that the $v$ in Ohm's Law is the potential across the resistor, not the potential of the resistor relative to an arbitrary ground. This is why very high voltages are used in power lines: $\Delta v << v$ (because $\Delta v$ and $v$ are measured using different reference points) and so this leads to very high efficiency compared to when higher currents and lower potentials are used.

\subsubsection{Heat}
Following on from the definition of power, the thermal energy released by the resistor in time $t = t_2 - t_1$ can be calculated by the following (thermodynamics states that all of the energy dissipated will eventually become heat):
\[W=\int_{t_1}^{t_2} p(t) \: dt = \int_{t_1}^{t_2} v(t)i(t) \: dt\]

Of-course, real materials have a limited amount of power which they can dissipate without becoming damaged. For resistors this is called the resistor power rating. Many resistors are typically rated at $1/10$, $1/8$ or $1/4$ Watts. Resistors with higher power ratings than this are called power resistors.

For similar reasons, resistors also have temperature ratings which depend upon the power loads on the resistors.

The resistance from the material will also change with temperature. This is described by the temperature co-efficient of resistance (TCR) which is expressed as the change in resistance (in parts per million) per degree Kelvin. This is not typically linear, instead being parabolic with temperature.

Sometimes resistance can change with potential as well and so there is a voltage rating and coefficient analogous to those for temperature. The voltage rating can be calculated as follows:
\[Rated Voltage(V)=\sqrt{Rated Power(W) \times Nominal Resistance Value(\Omega)}\]

\subsection{Noise in a Resistor}
In general thermal noise dominates at high frequencies and current noise at low frequencies.
\subsubsection{Thermal Noise}
Thermal noise is evenly distributed across all frequencies and is dependent only on the value of resistance and the resistor's temperature:
\[S_T = 4kTR\]
Where $k = 1.3807 \times 10^{-23} JK^{-1}$ (Boltzmann's constant). 

\subsubsection{Current Noise}
Current noise in a resistor does depend upon the material and the frequency:
\[S_E = C \, \frac{U^2}{f}\] where $C$ is a constant depending upon the material of the resistor and $U$ is the potential drop across the resistor.
This is often expressed in decibels:
\[[NI]_{dB} = 20 \log \left( \frac{u}{U} \times 10^6 \right)\]

\section{Capacitors}

\section{Polarisation}

\section{Electrets}

\chapter{Oscillations and Waves}

\chapter{Batteries and Fuel Cell}

\chapter{Appencices}
\section{Distrubution and Licensing}
This work is distrubuted under Creative Commons Attribution-ShareAlike 4.0
International (see https://creativecommons.org/by-sa/4.0/).

\end{document}
